\documentclass[../thesis.tex]{subfiles}

\begin{document}

\chapter{Conclusions}\label{chp:Discussion}
%Make link to initial research questions
\onehalfspacing
KESS2 funded this research \cite{KESS2023} in collaboration with the Clinical Futures \cite{UniAneurinBevanHealthBoardc}, within the Aneurin Bevan University Health Board (ABUHB). The aim of the project was to produce a decision support tool, supporting the mathematical modelling unit in bed and staffing resource requirements. The chapter serves as a summary of the research undertaken in this thesis. It provides a brief overview of the research questions listed in Section \ref{chp:Introduction} and the methods used to answer them. The chapter also presents contributions of the thesis, limitations, impact in practice and recommendations for future work.


\section{Research Summary}
Chapter \ref{chp:Introduction} provided an introduction to the frail and elderly population within Wales. The chapter discussed the demographic changes that are occurring within the population and the impact that these changes will have on the healthcare system. The types of hospitals and specialties which the health board currently have were also discussed. Current bed and staff planning methods were analysed with the following four research questions being introduced:


\begin{enumerate}
    \item How do the clinical and demographical attributes of frail and elderly patients effect their length of stay within hospital?    
    \item How best can specialties be organised among a network of hospitals to ensure staffing and bed costs are minimised, whilst still meeting the demand for frail and elderly patients?
    \item Can linking predictive and prescriptive analytics provide improvements for decision making for frail and elderly services?  
    \item How can deterministic and two-stage stochastic models be used to plan hospital services for frail and elderly patients within Aneurin Bevan University Health Board?
\end{enumerate}

Chapter \ref{chp:LiteratureReview} provided two literature reviews of the practice of Operational Research and Management Science (OR/MS) approaches in the planning of care for the frail and elderly. The underutilisation of OR/MS techniques, the absence of comprehensive holistic care planning, and the implications of increases in demand on healthcare systems have all been noted as gaps in current literature. Within this thesis, these gaps were addressed.

Chapter \ref{chp:predictive} addressed the theory underlying the most popular predictive analytical techniques presently used in healthcare. The results demonstrated the benefit of utilising more complex models, such as classification and regression trees (CART), instead of simpler models, such as linear regression to predict the length of stay (LOS) of frail and elderly patients. These results yielded a more accurate prediction of LOS, which is important for planning purposes. A step by step practical example was also included so that healthcare professionals could quickly apply these strategies to their own departments and data. To enable model adaptation and parameter optimisation, detailed executable Python code was provided.

Chapter \ref{chp:presciptive} provided an introduction to two prescriptive methodologies, deterministic and two-stage stochastic modelling. Expanding on the two-stage stochastic programming paradigm and building on the tests introduced by Maggioni and Wallace \cite{Maggioni2010}, this chapter went further by creating two-dimensional decision variables which are dependent on each other along with the application to a different field of research, namely frail and elderly patient planning. The equations generated allow for the optimisation of the number of beds and staff required to meet demand. The models created were robust in terms of working ability. Furthermore, the modelling was shared with ABUHB, especially how the models work. The tests discussed in \cite{Maggioni2010}, have also been employed, applied and evaluated to each of the examples. 


Chapter \ref{chp:Experimental Analysis} presented the findings of the predictive and prescriptive analytical models. Section \ref{sec:dataintroduction} provided an overview of the current data and trends within ABUHB and within the frail and elderly community. Section \ref{sec:predictiveresults} aimed to answer the first research question by generating CART models to predict LOS of frail and elderly patients. The models also compared the impact of frailty on LOS. The results highlighted the improved R$^{2}$ and accuracy scores when using CART models over traditional linear and logistic regression methods. These CART models also enabled patient groupings of similar attributes to be determined. Section \ref{sec:prescriptiveresults} aimed to answer the second research question by applying the deterministic and two-stage stochastic models generated in Chapter \ref{chp:presciptive} to ABUHB data. The models determined how beds should be planned and staff deployed based on figures from Public Health Scotland and NHS Jobs, to ensure costings are minimised. Results showed the benefits of utilising the two-stage stochastic model to plan their beds and staff over traditional deterministic models. Any savings made by the NHS could be reinvested into other areas of healthcare.

Chapter \ref{chp:Linking} discussed how predictive and prescriptive analytics could be used in combination for efficiently planning hospital specialty beds and staffing requirements for a network of hospitals in South East Wales. Research questions three and four were answered by comparing the CART results to the traditional averages. The primary aim of employing CART models was to explore the potential benefits of using prescriptive methods for resource capacity planning in the healthcare context. By using CART models, we aimed to gain a more comprehensive understanding of the factors influencing LOS and its variations among patients. The predictive capabilities of CART allowed us to identify non-linear relationships and interactions among various patient characteristics and medical conditions, leading to more accurate and individualised LOS predictions. The derived daily bed demand, informed by the CART predicted LOS, provided a more realistic representation of the variation within hospital LOS. Unlike traditional averages, which might overlook patient-specific factors affecting LOS, the CART-based approach captured a wider range of LOS variations, reflecting the diversity and complexity of patient care requirements.

Finally, Chapter \ref{chp:tool} provided a tutorial on how to use the deterministic and two-stage stochastic models generated. The models were implemented in Microsoft Excel using the OpenSolver add-in, and in Python using the PuLP package. Both versions of the models were included to reach a wider audience and subsequently uploaded to GitHub for future use. These models are available from \cite{Williams2023}. These tutorials aim to provide a step by step guide on how to use the models and be applied to other healthcare organisations. As future patient demographics change, the models can be rerun with updated data to determine the most efficient way to plan beds and staff.

\section{Research Contributions}
The findings presented within this thesis have provided a number of novel contributions to the literature on OR and healthcare applications. These contributions are as follows:

\begin{itemize}
    \item The literature reviews presented in Chapter \ref{chp:LiteratureReview} provided a comprehensive overview of the current literature on frail and elderly care planning with OR/MS methods and hierarchical prediction models to predict LOS. This allowed themes and methods to be identified and enabled gaps within the literature to be determined. The reviews focused specifically on frail and elderly patients, showing the limited research published within this area.
    \item The development of the predictive models (Chapter \ref{chp:predictive}) provided a novel method to predict LOS of frail and elderly patients, instead of considering these patients within the adult population. This allowed for the impact of frailty on LOS to be determined. This chapter has used sophisticated techniques which are underutilised within the context of healthcare.
    \item Prescriptive models were developed to plan beds and staff for frail and elderly patients (Chapter \ref{chp:presciptive}). These models expanded upon the work of Maggioni and Wallace \cite{Maggioni2010}, by applying to the area of healthcare OR and analysing bed and staff requirements.
    Instead of planning on a ward by ward basis, these models enabled holistic planning to take place across the health board.
    \item By linking predictive and prescriptive analytics, decision-makers can achieve a more comprehensive view of their data and use it to make more informed decisions. Chapter \ref{chp:Linking} demonstrated how these methods could be linked, providing a number of examples of different methods. This allowed for scenario analysis to be performed, using a combination of techniques to provide unique insights into the ABUHB healthcare system.
\end{itemize}

\section{Limitations of the Study}
There are several limitations to this research. Firstly, the reliance on historical activity data to predict future demands poses a significant constraint. One of the primary concerns is the potential omission of unmet demand from the dataset. Activity data typically capture the services that have been provided and recorded in the system, but they may not fully represent the actual demand for healthcare services. Unmet demand, or the demand that goes unaddressed due to capacity constraints or other factors, is critical to consider in resource allocation to ensure the system can meet the true needs of the population. Another challenge is the inclusion of LOS which reflect poor historic system performance. When historical LOS's are incorporated into the model, they may inadvertently perpetuate inefficiencies or suboptimal practices from the past.

Another limitation of the study is its reliance on pre-Covid-19 data for capacity allocation modelling. As the Covid-19 pandemic has had a profound and unprecedented impact on healthcare systems worldwide, using data prior to the pandemic might not fully reflect the current and future resource allocation needs. The pandemic has introduced unique challenges, such as surges in patient volumes, changes in patient acuity, and shifts in healthcare priorities. The demand for resources, including beds, nursing staff, and other medical supplies, has been substantially affected during this period.

A notable limitation of the model is its omission of ward sizes and the number of beds or staff per ward in the resource allocation process. Instead, the model adopts a more holistic approach, considering the healthcare facility as a whole entity. While this simplification may offer practicality and ease of implementation, it overlooks crucial ward-level variations in patient capacity and staffing requirements.

This research relied on the use of open source data from StatsWales \cite{StatsWalespp1} and Public Health Scotland \cite{PHS2021}. Therefore, there was potentially inaccurate or imprecise data to populate the model. The accuracy of the model's predictions heavily relies on the quality and reliability of the data used as input.

Finally, both the Excel and Python tools, utilise various constraints and objective functions that were presented in Sections \ref{sec:DeterministicModel} and \ref{sec:twostagestochasticmodel} respectively. Whilst other limitations can be alleviated by changing the data within the Excel worksheets and Python scripts, if new constraints were needed to be added, this would require the users to have knowledge and understanding of the formulation of the mathematical constraints.

\section{Impact in Practice}
This research collaboration with the Clinical Futures team at ABUHB has significantly influenced the development and direction of the project. ABUHB's substantial time and financial investment in the research reflect their interest in deriving benefits from the outcomes. At the time when the thesis was finished, the planning team were aiming to share the results with the executive board using an SBAR. This may lead to further support the Aneurin Bevan Continuous Improvement team (ABCi), as the project continues with the support of the interim director of planning and the lead of the mathematical modelling unit. They have shown interest in the model's potential and are expected to utilise it to craft a compelling case study for senior decision-makers within the health board.

Moreover, the team at ABCi offers an engaging analytics program, providing interactive training to front-line staff in data analysis and mathematical modelling techniques \cite{abcimodelling,Gartner2022a}. This presents an exciting opportunity to integrate these models into their program, ensuring broader dissemination and impactful utilisation of the research findings. By being part of their curriculum, these models have the potential to empower healthcare professionals with valuable insights, ultimately driving informed decision-making and resource optimisation within the healthcare domain.


\section{Future Work}
The work presented in this research has provided a number of insights into the ABUHB healthcare system, however, there are a number of areas that could be further explored. The following areas for future study were identified:
\begin{itemize}
    \item Chapter \ref{chp:predictive} and Chapter \ref{chp:Experimental Analysis}: The predictive models presented within this thesis could be further developed to include other attributes detailing a patient's medical history. These may include the number of previous admissions, the number of previous admissions to the same ward and the number of previous admissions to the same specialty.
    \item Chapter \ref{chp:presciptive} and Chapter \ref{chp:Experimental Analysis}: The prescriptive models presented within this thesis could be further developed to include additional variables. Further work could include planning specialties by specific wards rather than generalising across specialties. Additionally, the demand for resources within the hospital such as phlebotomists, radiographers and physiotherapists could be included. 
    \item Chapter \ref{chp:Experimental Analysis} and Chapter \ref{chp:Linking}: The models were developed using either three years' worth of data or splitting the data by year. This could be further developed by using a time constraint to plan on specific time periods rather than on a longer-term time scale. This would create a more dynamic model where the health board would be able to adapt to seasonal demand changes or determine how beds and nursing resources would change on a smaller time scale.
    \item Chapter \ref{chp:Linking}: The linked predictive and prescriptive models presented within this thesis could use sampling from the end nodes rather than using the average. This would provide a randomised solution to the problem, which could be used for prediction purposes. Further investigation into population predictions could also be used within the models as a separate input. This chapter also investigated a range of various scenarios, including the addition of the Grange University Hospital (GUH). Due to the limitation of the data received, i.e., pre-2020, the impact of the new hospital was unable to be determined, as this opened in 2021, however, using the data prior to its opening, GUH could still be investigated. Therefore, the model still provided useful results and recommendations for bed planning and nursing staff. To determine how beds should be planned, more recent data should be used and the effects on demand following the Covid-19 pandemic can be visualised. The model could be developed further to have real time updates of the demand entering the system so planning can be conducted on a more operational scale.
 
\end{itemize}

Finally, further research could consider analysing other areas of the ABUHB healthcare system, as well as other age groups.
\end{document}
