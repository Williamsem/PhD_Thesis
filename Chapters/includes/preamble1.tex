\usepackage[a4paper, top=25.4mm, bottom=20.5mm, right=20.5mm, left=40mm]{geometry}

%\usepackage[a4paper, top=2.5cm, bottom=2.5cm, right=2.5cm, left=2.5cm]{geometry} 

\usepackage{subfiles}           % allows each chapter to be its own seperate, individually
                                % compileable file

\usepackage[final]{graphicx}    % provides \includegraphics
                                % final option provided to override global "draft" argument
                                % and always print figures. if you want to produce a version
                                % with only dummy figures, remove the final option
%\usepackage{biblatex}
%\usepackage{bibtex}

\usepackage[numbers,sort]{natbib}
\usepackage{parskip}
\usepackage{subcaption}
\usepackage{setspace}
\usepackage{longtable}
\usepackage{multirow}
\usepackage{mathtools}
\usepackage[normalem]{ulem}
\usepackage{soul}
\usepackage[dvipsnames]{xcolor}
\usepackage{tikz}
\usetikzlibrary{arrows}
\usetikzlibrary{decorations.markings}
\usetikzlibrary{calc}
\usetikzlibrary{shapes, positioning}
\usetikzlibrary{shapes.geometric}
\usetikzlibrary{patterns,snakes}
\usetikzlibrary{decorations.text}
\usetikzlibrary{arrows.meta}
\usepackage{tikz}
\usetikzlibrary{mindmap}
\usetikzlibrary{shapes.geometric,arrows}
\tikzstyle{startstop} = [rectangle, rounded corners, minimum width=3cm, minimum height=1cm,text centered, draw=black, fill=red!30]

\tikzstyle{heading} = [rectangle, minimum width=9cm, minimum height=1cm,text centered, draw=black,fill=blue!30]

\tikzstyle{io} = [trapezium, trapezium left angle=70, trapezium right angle=110, minimum width=3cm, minimum height=1cm, text centered, draw=black, fill=blue!30]
\tikzstyle{process} = [rectangle, minimum width=3cm, minimum height=1cm, text centered, draw=black, fill=orange!30]
\tikzstyle{term} = [rectangle, minimum width=3cm, minimum height=1cm, text centered, draw=black, fill=blue!30]
\tikzstyle{term1} = [rectangle, minimum width=3cm, minimum height=1cm, text centered, draw=black, fill=green!30]
\tikzstyle{decision} = [diamond, minimum width=3cm, minimum height=1cm, text centered, draw=black, fill=green!30]
\tikzstyle{arrow} = [thick,->,>=stealth]
\usepackage{pdflscape}
\usepackage{standalone}
\usepackage{float}
\usepackage{caption}
\usepackage{subcaption}
\usepackage{amsmath}
\usepackage{amsfonts}
%\usepackage{algpseudocode}
\usepackage{amsthm}
\usepackage[ruled]{algorithm2e}
\usepackage{rotating,tabularx}
\usepackage{adjustbox}
\usepackage{enumerate}

%\usepackage{breakurl}
\PassOptionsToPackage{hyphens}{url}\usepackage{hyperref}
\usepackage{url}	
%\hypersetup{breaklinks=true}
%\usepackage{hyperref}
\usepackage[utf8]{inputenc}
\usepackage[toc]{appendix}
\usepackage{minted}
\usepackage{array}
\usepackage{booktabs}
\usepackage{siunitx}
\usepackage{multicol}
\usepackage{multirow}
\usepackage[toc]{appendix}
\usepackage{colortbl}

\newcommand{\repeatcaption}[2]{%
  \renewcommand{\thefigure}{\ref{#1}}%
  \captionsetup{list=no}%
  \caption{#2 (repeated from page \pageref{#1})}%
  \addtocounter{figure}{-1}% So that next figure after the repeat gets the right number.
}

\usepackage{txfonts}
\usepackage{lmodern}
\usepackage{pifont}% http://ctan.org/pkg/pifont
%\hypersetup{breaklinks=true}

% Package to manage headers and footers
\usepackage{fancyhdr}

% define two header-footer styles
% 'normal' for most pages
\fancypagestyle{normal}{
\fancyhf{}
\fancyhead[L]{\small{\slshape \leftmark}} %chapter
\fancyhead[R]{\thepage} %footer
\renewcommand{\headrulewidth}{1pt}
}
% 'chapterstyle' for start of chapters (no chapter in header)
\fancypagestyle{chapterstyle}{
    \fancyhf{}
    \fancyhead[R]{\thepage}
    \renewcommand{\headrulewidth}{0pt}% Line at the header invisible
}
\fancypagestyle{none}{
    \fancyhf{}
    \renewcommand{\headrulewidth}{0pt}
}

% Package to automatically change headerfooter style at start of chapters
\usepackage{etoolbox}
\patchcmd{\chapter}{\thispagestyle{plain}}{\thispagestyle{chapterstyle}}{}{}

\linespread{1.7}

\newtheorem{definition}{Definition}
\newtheorem{theorem}{Theorem}
\newtheorem{lemma}{Lemma}
\newtheorem{conjecture}{Conjecture}
\newtheorem{corollary}{Corollary}
\definecolor{bg}{RGB}{249, 251, 233}

\setcounter{tocdepth}{1}
\setcounter{secnumdepth}{4}

\def\Vhrulefill{\leavevmode\leaders\hrule height 0.7ex depth \dimexpr0.4pt-0.7ex\hfill\kern0pt}


\newcommand{\specialcell}[2][c]{%
  \begin{tabular}[#1]{@{}c@{}}#2\end{tabular}}

\newcommand{\specialcelll}[2][l]{%
  \begin{tabular}[#1]{@{}l@{}}#2\end{tabular}}

\newcommand{\specialcellr}[2][r]{%
  \begin{tabular}[#1]{@{}c@{}}#2\end{tabular}}


\newcommand{\cmark}{\ding{51}}%
\newcommand{\xmark}{\ding{55}}%


\newcommand*{\signatureblock}{%
    % setup a a signature block for the declarations page
\begin{tabular}{@{}lp{50mm}@{}llp{20mm}}
    Signed & \hrule & (candidate) & Date & \hrule\\
\end{tabular}

}

\usepackage{listings}
\usepackage{xcolor}

\definecolor{codegreen}{rgb}{0,0.6,0}
\definecolor{codegray}{rgb}{0.5,0.5,0.5}
\definecolor{codepurple}{rgb}{0.58,0,0.82}
\definecolor{backcolour}{rgb}{0.95,0.95,0.92}

\lstdefinestyle{mystyle}{
    backgroundcolor=\color{backcolour},   
    commentstyle=\color{codegreen},
    keywordstyle=\color{magenta},
    numberstyle=\tiny\color{codegray},
    stringstyle=\color{codepurple},
    basicstyle=\ttfamily\footnotesize,
    breakatwhitespace=false,         
    breaklines=true,                 
    captionpos=b,    
    language = python,
    keepspaces=true,                 
    numbers=left,                    
    numbersep=5pt,                  
    showspaces=false,                
    showstringspaces=false,
    showtabs=false,                  
    tabsize=2
}

\lstset{style=mystyle,literate={~}{{$\sim$}}1}


\usepackage{blkarray}% http://ctan.org/pkg/blkarray
\newcommand{\matindex}[1]{\mbox{\scriptsize#1}}% Matrix ind

\usepackage{mathrsfs}

\newcommand{\subsubsubsection}[1]{\paragraph{#1}\mbox{}\\}
\setcounter{secnumdepth}{4}
\setcounter{tocdepth}{4}

\newcommand\MyBox[2]{
  \fbox{\lower0.75cm
    \vbox to 1.7cm{\vfil
      \hbox to 1.7cm{\hfil\parbox{1.4cm}{#1\\#2}\hfil}
      \vfil}%
  }%
}

\usepackage{highlight}
\newcommand{\g}[1]{\gradientcelld{#1}{0.29}{0.32}{0.35}{red}{yellow}{green}{60}}
\newcommand{\ga}[1]{\gradientcelld{#1}{8.3}{14}{31}{green}{yellow}{red}{60}}

\newcommand{\ac}[1]{\gradientcelld{#1}{0.884}{0.893}{0.9}{red}{yellow}{green}{60}}
\newcommand{\pa}[1]{\gradientcelld{#1}{0.847}{0.86}{0.9}{red}{yellow}{green}{60}}
\newcommand{\re}[1]{\gradientcelld{#1}{0.875}{0.89}{0.92}{red}{yellow}{green}{60}}
\newcommand{\ti}[1]{\gradientcelld{#1}{7.8}{14}{25.5}{green}{yellow}{red}{60}}

\newcolumntype{x}[1]{>{\centering\arraybackslash\hspace{0pt}}p{#1}}

\setcounter{MaxMatrixCols}{20}

\usepackage{fancyvrb}
\usepackage{alltt}

\usepackage{placeins}

