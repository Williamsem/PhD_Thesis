\documentclass[thesis.tex]{subfiles}

\begin{document}
\chapter*{Abstract}
 \addcontentsline{toc}{chapter}{Abstract}
The National Health Service in the United Kingdom is under increasing pressure to provide and deliver high-quality care to an ageing population with complex health needs. This research project was funded by KESS2 in collaboration with the Aneurin Bevan University Health Board in South East Wales.


This thesis investigates the potential of using predictive and prescriptive analytics to optimise bed capacities and staffing requirements within Aneurin Bevan University Health Board. Homogeneous patient clusters are identified through the use of classification and regression trees to predict the length of stay of the frail and elderly population. Deterministic and two-stage stochastic optimisation models are developed to determine bed capacities and staffing requirements, taking into account factors such as patient acuity, length of stay, and resource constraints. 

The predictive and prescriptive models are then combined by using the classification and regression tree models to determine demand values to be inputted into the deterministic and two-stage stochastic models. To determine the benefit and cost savings of using the stochastic implementation over traditional deterministic models, the value of the stochastic solution is calculated.

Through the application of scenario analysis, the methods allow various case studies to be modelled to provide insights into how the system would cope with fluctuations in resources, demand or organisational changes. The findings of this thesis have important implications for healthcare providers and policymakers, highlighting the potential for the combination of predictive and prescriptive analytics to improve the quality and efficiency of healthcare delivery. The study also provides a framework for future research in this area, including the potential for applying these techniques to other healthcare settings and populations.

% \cleardoublepage
% \thispagestyle{empty}

\end{document}
